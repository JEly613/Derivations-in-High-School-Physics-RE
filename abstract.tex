\chapter*{Preface}

\qquad Oftentimes, equations are given to students on various equation sheets on physics tests. Although knowing the equations is important, it is even more important to know the derivations of the equations as it provides a deeper understanding of the physical concepts behind the equations and builds physical intuition required to solve problems. Using the mathematical flexibility gained through proving formulas, students will better be able to apply their knowledge to solve more complex physics problems.
\par These notes attempt to derive as many formulae and concepts as possible that pertain to high school physics, from AP courses to Physics Olympiads. Equations featured in AP Physics 1 and 2 are derived without calculus (as much as possible), while calculus is employed when necessary to derive more difficult expressions. Vectors are represented using boldface ($\textbf{v}$) in this text, and, a star ($\bigstar$) is used to mark the conclusion of each concept/derivation.
\par The sections required for various tests are marked with the following colors:
\begin{itemize}
\color{OrangeRed}
\item{$\blacktriangleright$} AP Physics 1
\color{RoyalBlue}
\item{$\blacktriangleright$} AP Physics 2
\color{PineGreen}
\item{$\blacktriangleright$} AP Physics C
\color{Goldenrod}
\item{$\blacktriangleright$} F=ma Exam
\color{Orchid}
\item{$\blacktriangleright$} USAPhO Exam
\end{itemize}

\mbox{}\linebreak
