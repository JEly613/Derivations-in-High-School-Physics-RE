\chapter{Thermodynamics}\label{cap5}

\section{\color{Orchid} $\blacktriangleright$ \color{black} Root Mean Square Velocity}
An ideal gas can be represented by a Gaussian distribution of the velocity of the singular particles in the gas, given by the following distribution:
\begin{equation}
    f(v) = 4\pi\left[\frac{m}{2\pi k_BT}\right]^{3/2}v^2e^{\frac{-mv^2}{2k_BT}}
\end{equation}
\noindent The equation is a bit messy, but it's important to recognize what it means. The function is a \textit{Probability Distribution Function (PDF)}, which is a special type of function that describes the way in which a random system is arranged. First of all, the equation can be broken into two parts like so
\begin{align*}
    f(v) &= 4\pi v^2\left[\frac{m}{2\pi k_BT}\right]^{3/2} \\
    &\cdot e^{\frac{-mv^2}{2k_BT}}
\end{align*}
\noindent The lower part 